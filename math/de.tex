% -*-latex-*-
% de.tex
%
% Differential Equations
%

\begin{section}{Differential Equations}
  % existance:
  %  if y'=f(x,y) is defined
  % uniqueness
  %  there is exctly one solution y(a)=b
  % intersection
  %  integral curves do not intersect
  % autonomous
  \begin{subsection}{Forms}
  \begin{tabular}{l l l l}
    ODE & Homogeneous & Particular & Description\\
    & Solution & Solution & \\
    \hline \\
    $A(t)\dot{x}+B(t)x=0$ & $x_h$ & & Homogeneous ODE\\
    $A(t)\dot{x}+B(t)x=q(t)$ & & $x_p$ & Inhomogeneous ODE $q(t) \ne 0$\\
    $\dot{x}=ax$ & $x(t)=Ce^{at}$ & & Exponential Growth (Autonomous ODE)\\
    $\dot{x}=f(x) \cdot ax$ & $$ & & Logistic Population Model\\
    $\dot{x}+k\,x=q(t)$ & $C\,e^{-kt}$ & $e^{-kt}\int e^{kt} q(t) dt$ &
      Constant Coefficient First Order ODE\\
    $\dot{x}+k\,x=B\,e^{at}$ & $C\,e^{-kt}$ & $\frac{B\,e^{at}}{k+a}$ &
      Exponential Response\\
    $m\,\ddot{x}+k\,x=0$ & $$ & $$ & Simple Harmonic Oscillator\\
    $m\,\ddot{x}+b\ddot{x}+k\,x=0$ & $$ & $$ & Damped Harmonic Oscillator\\
%    $$ & $$ & $$ & \\
  \end{tabular}
  \end{subsection}

  \begin{subsection}{Superposition}
    If $x_1$ is a solution to $DE=q_1(t)$ and $x_2$ is
    a solution to $DE=q_2(t)$ then for any constants $a$
    and $b$, $a\,x_1 + b\,x_2$ is a solution to
    $DE=a\,q_1(t) + b\,q_2(t)$.
  \end{subsection}

  \begin{subsection}{General Solution}
    The general solution for a first order ODE is $x_p(t) + c\,x_h(t)$
    for any constant $c$. For an ODE of any order, the number of
    constants is equal to the order.
  \end{subsection}

  \begin{subsection}{Separation of Variables}
    \begin{align*}
      \frac{dy}{dx} &= f(x)g(y) \\
      \frac{dy}{g(y)} &= f(x)\,dx \\
      \int{\frac{dy}{g(y)}} &= \int{f(x)}\,dx+c \\
    \end{align*}
  \end{subsection}
  \begin{subsection}{Integrating Factors}
    \begin{align*}
      \frac{dy}{dx} + P(x)y &= Q(x) \\
      \frac{d}{dx}(e^{\int\,P(x)\,dx}\,y) &= Q(x)\,e^{\int\,P(x)\,dx} \\
      \int{\frac{d}{dx}(e^{\int\,P(x)\,dx}\,y)} &= \int{Q(x)\,e^{\int\,P(x)\,dx}} \\
      e^{\int\,P(x)\,dx}\,y &= \int{Q(x)\,e^{\int\,P(x)\,dx}} \\
      y &= e^{-\int\,P(x)\,dx}\,\int{Q(x)\,e^{\int\,P(x)\,dx}} \\
    \end{align*}
  \end{subsection}
  \begin{subsection}{Differential Operators}
    \begin{align*}
      a_n &= \text{Constant coefficients} \\
      \sum_{n=0}^{N} a_n\,y^{(n)} &= f(t) && \text{Linear $N^{th}$ order ODE} \\
      \sum_{n=0}^{N} a_n\,D^{n} &= p(D) && \text{Differential Operator $D$} \\
      p(D)\,y &= f(t) && \text{Operator form in $D$} \\
      (p(D) + q(D))y &= p(D)y + q(D)y && \text{Summation} \\
      p(D)(c_1\,f + c_2\,g) &=
      c_1\,p(D)\,f +  c_2\,p(D)\,g && \text{Linearity} \\
      (p(D)\,q(D))y &= p(D)\,q(D)y && \text{Multiplication} \\
      p(D)\,q(D)y &= q(D)\,p(D)y && \text{Multiplication commutes} \\
      p(D)e^{at} &= p(a)\,e^{at} && \text{Substitution} \\
      p(D)e^{at}\,y &= e^{at}p(D+a)\,y && \text{Exponential Shift} \\
      p(D)\,y(t-c) &= f(t-c) && \text{Time Invariance} \\
    \end{align*}
  \end{subsection}
  \begin{subsection}{Exponential Response}
     Particular Solutions
     \begin{align*}
       p(D)\,y &= e^{at} && \text{Exponential Response in operator form} \\
       y &= e^{at}/p(a) && \text{if $p(a) \ne 0$} \\
       y &= t\,e^{at}/\dot{p}(a) &&
           \text{if $p(a)=0 \land \dot{p}(a) \ne 0$} \\
       y &= t^{2}\,e^{at}/\ddot{p}(a) &&
           \text{if $p(a)=0 \land \dot{p}(a)=0 \land \ddot{p}(a) \ne 0$} \\
       y &= t^{s}\,e^{at}/p^{(s)}(a) &&
           \text{if $p^{(s-1)..0}(a)=0 \land p^{(s)}(a) \ne 0$} \\
     \end{align*}
  \end{subsection}
  \begin{subsection}{Generalized Characteristic Equation}
  \begin{align*}
    N &= \text{Order of equation}\\
    n &= \text{Index of order}\\
    M &= \text{Number of distinct roots}\\
    m &= \text{Index of first distinct root in $n$ space}\\
    a_n &= n^{th} \text{ODE constant coefficient}\\
    K &= \text{Number of repeated roots}\\
    k &= \text{Index of repeated root}\\
    r_n &= \text{$n^{th}$ (sorted) root}\\
    c_n &= \text{$n^{th}$ solution constant}\\
    \text{ODE} &= \sum_{n=0}^{N} a_n\,y^{(n)} && \text{Linear $N^{th}$ order ODE} \\
    \text{CP} &=\sum_{n=0}^{N} a_nr^n && \text{Characteristic Polynomial}\\
    r_{n} &= solve(\text{CP} = 0, r) && \text{Roots of CP}\\
    \text{CE} &=
       \sum_{m=0}^{M} e^{r_{m}\,x} \left( \sum_{0}^{K-1} c_{m+k} x^{k-1} \right) &&
       \text{Characteristic Equation}\\
  \end{align*}
  \end{subsection}
  \begin{subsection}{Step and Delta Functions}
    \begin{align*}
      u(t) &= (t<0) \,?\, 0 : 1 && \text{Unit step function}\\
      u(t-a) &= (t<a) \,?\, 0 : 1 && \text{Unit step function at $a$}\\
      u(0) &= \frac{1}{2} && \text{Heaviside step function}\\
      u_{ab}(t) &= u(t-a) - u(t-b) && \text{Box function $a<t<b$}\\
      \delta(t) &= (t=0)\,?\,\infty : 0 && \text{Dirac delta function}\\
      f(t)\delta(t) &= f(0)\delta(t) \\
      f(t)\delta(t-a) &= f(a)\delta(t) \\
      \frac{d}{dt}u(t) &= \delta(t)\\
      \int_c^d \delta(t)\,dt &= (c<0<d)\,?\,1:0 \\
      \int_c^d f(t)\delta(t)\,dt &= (c<0<d)\,?\,f(0):0 \\
      \int_c^d f(t)\delta(t-a)\,dt &= (c<a<d)\,?\,f(a):0 \\
    \end{align*}
  \end{subsection}

  \begin{subsection}{Impulse Response}
    The solution of an impulse response is a homogeneous solution with
    {\em almost} rest initial conditions as shown below.
    \begin{align*}
       p(D)\,y &= e^{at} && \text{Exponential Response in operator form} \\
       y &= e^{at}/p(a) && \text{if $p(a) \ne 0$} \\
      \sum_{n=0}^{N} a_n\,x^{(n)} &= \delta(t) && && \text{$N^{th}$ order ODE} \\
      x^{(n)}(0) &= 0 && n \ne N-1 && \text{Almost all at rest} \\
      x^{(n-1)}(0) &= 1/a_n && n = N-1 && \text{Except for next-to-last} \\
    \end{align*}
  \end{subsection}

  \begin{subsection}{Convolution}
    \begin{align*}
      f(x) \ast g(x) &= \int_{0-}^{t+} f(\tau)\,g(t-\tau)\,d\tau &&
      \text{Single Sided}\\
      (c_1f_1 + c_2f_2) \ast g &= c_1(f_1 \ast g) + c_2(f_2 \ast g) &&
      \text{Distributive}\\
      f \ast g &= g \ast f && \text{Communitive} \\
      f \ast (g \ast h) &= (f \ast g) \ast h && \text{Associative} \\
      \delta(t) \ast f(t) &= f(t) && \text{Multplicitive identity} \\
      \delta(t-a) \ast f(t) &= f(t-a) && \text{Shift} \\
    \end{align*}
  \end{subsection}

  \begin{subsection}{Green's Formula}
    The solution for an arbitrary response is the convolution of the response
    $f(t)$ with the impulse response $w(t)$.
    \begin{align*}
      p(D)\,y &= f(t) && \text{Arbitrary Response in operator form} \\
      y &= (f \ast w)(t) && \text{Solution using convolution} \\
    \end{align*}
  \end{subsection}

  \begin{subsection}{Linearization}
    \begin{align*}
      \dot{x} &= ax + by && \text{System 1}\\
      \dot{y} &= cx + dy && \text{System 2}\\
      \begin{pmatrix}
        \dot{x} \\
        \dot{y} \\
      \end{pmatrix}
      &=
      \begin{pmatrix}
        a & b \\
        c & d \\
      \end{pmatrix}
      \begin{pmatrix}
        x \\
        y \\
      \end{pmatrix}
      && \text{System in matrix form}\\
      A &=
      \begin{pmatrix}
        a & b \\
        c & d \\
      \end{pmatrix}
      && \text{System coefficients}\\
      \lambda &= \lambda_1 , \lambda_2 && \text{Eigenvalues}\\
      A\,\mathbf{v} &= \lambda\,\mathbf{v} &&  \text{Eigenvectors}\\
      \mathbf{a} &=
      \begin{pmatrix}
        h \\
        k \\
      \end{pmatrix}
      && \text{Solution coefficients}\\
     \dot{\mathbf{x}} &= A\,\mathbf{x} && \text{System in compact form} \\
     \mathbf{x} &= e^{\lambda\,t}\,\mathbf{a}
     && \text{Solution in compact form} \\
     \lambda^2 - (a+d)\lambda + (ad - bc) &= 0
     && \text{Characteristic Equation (CE)} \\
     \lambda^2 - tr(A)\lambda + det(A) &= 0
     && \text{CE with trace and determinant} \\
    \end{align*}
  \end{subsection}
\end{section}
